% !TEX root = bipartite.tex

\section{Related work}

One popular approach when analyzing two-mode networks is to transform them into
one-mode networks through a method known as projection and then use the methods
available for social networks. \citet{newman2001scientific1} and
\citet{barabasi2002evolution} study scientific collaboration networks between
authors of scientific papers by projecting the authoring network (author-paper)
to an unweighted co-authoring network (author-author). A tie is defined between
two authors if they have at least one collaboration together, without taking
into account the frequency of collaboration between the authors. In order to
avoid the loss of information through unweighted projection, some authors
proposed a weighting method. \citet{ramasco2006social} and
\citet{li2007evolving} use a simple mechanism to assign weights, based on the
number of times those authors collaborated. Other papers propose a different
approach to weighting. In \citet{newman2001scientific2} the contribution of a
co-authored article to the weight of an edge between two authors depends on the
degree of the article. This is based on the assumption that a low degree article
defines a stronger relationship between authors than a high degree one.
\citet{li2005weighted} considers a saturating effect where the increase in
strength of a relationship between two authors slows down when more articles
are added, as they already know each other well after writing some papers
together.

In order to avoid loss of information through projection (weighted or
unweighted), several features and methods have been proposed for bipartite
networks, most of them borrowed from social networks.
\citet{borgatti1997network} studies visualization techniques of bipartite
networks and defines several properties for these type of networks like density
and centrality. \citet{latapy2008basic} uses several statistics to analyze
two-mode data and proposes new bipartite properties like clustering coefficient
and redundancy coefficient. \citet{lind2005cycles} and
\citet{zhang2008clustering} redefine the clustering coefficient for bipartite
networks by considering squares (4-cycles) instead of triangles, which are not
possible in a two-mode configuration. \citet{opsahl2011triadic} proposes a new
measure of clustering based on the notion of triadic closure defined between
three nodes of the same type. Several community detection methods have been
proposed for bipartite networks \citep{fortunato2010community}. However most of
these methods are not optimized for large networks on the scale seen in web or
e-commerce environment. Therefore when detecting communities in bipartite
networks a widely used approach is to project them on one of the node sets and
apply scalable community detection methods from social networks on the
projection.

The research work mentioned above focused either on specific networks like
collaboration network or generally on bipartite networks. They didn't consider
user-object networks as a particular class of bipartite networks. On the other
side, especially in a web or online environment user-object networks with their
particularities are becoming increasingly popular in modeling the interaction
between users (e.g. customers, listeners, watchers etc) and the online system
(e.g. e-commerce, music, video etc). \citet{huang2007analyzing} is analyzing the
bipartite consumer-product graphs representing sales transactions in an
e-commerce setting. They found a larger than expected average path length and
tendency of customers to cluster according to their purchases. A new
recommendation algorithm is proposed based on these findings.
\citet{grujic2009mixing} uses a bipartite representation of interactions between
users and web databases to study patterns of clustering based on users common
interests. They found a power law degree distribution of objects and a
disassortative mixing pattern with high degree (active) users interacting mostly
with low degree (unpopular) objects and low degree (inactive) users interacting
mostly with high degree objects (popular). The authors applied a spectral
clustering method on the weighted projected network and found communities
relevant to subjects of common interests.

Recently, \citet{shang10empirical} did an empirical analysis of two web-based
user-object networks collected from two large-scale web sites and found the same
power law degree distribution for objects and disassortative mixing pattern. A
new property is proposed called collaborative similarity capturing the diversity
of tastes based on the collaborative selection. For the lower-degree objects the
authors found a negative correlation between object collaborative similarity
(how similar are the users interacting with a specific object) and the object
degree. Therefore unpopular (low degree) objects are considered a good indicator
for the users common interests, while popular objects are less relevant.
Starting from this observation we propose a new weighting scheme for user-object
bipartite networks that will increase the relevance of unpopular object in the
network and decrease the importance of popular ones.
