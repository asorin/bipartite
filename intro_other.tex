%!TEX root = bipartite.tex

\section{Introduction}


\citet{klink06biblio} have shown...

Community finding algorithms are a popular area of study \citep{palla05nature,fortunato10review}.

Then the community membership of the churners for months five and six were examined - see Table \ref{tab:rate1} and also Figure \ref{fig:typical}. The most striking thing to note is that the churn rate among unassigned nodes is an order of magnitude lower than that of nodes assigned to communities. 
Note that, while the MOSES algorithm will often assign nodes to multiple communities, it can also leave some nodes ``unassigned'' (\ie not assigned to any community at all).

\begin{equation}
\label{formula:EditQuality}
   \alpha_{edit}(v_{i},v_{j})= \frac{d(v_{i-1},v_{j})-d(v_{i},v_{j})}{d(v_{i-1},v_{i})} %\nonumber
\end{equation}

We considered three sets of community finding experiments:
\begin{enumerate}
\item Applying the MOSES algorithm to each individual monthly time step graph.
\item Applying the MOSES algorithm to the combined four month graph. This binary unweighted graph was created by aggregating the sets of nodes and edges present in the individual time step graphs.
\item Dynamic community finding on the four time step graphs. 
\end{enumerate}

Then the community membership of the churners for months five and six were examined - see Table \ref{tab:rate1} and also Figure \ref{fig:typical}. The most striking thing to note is that the churn rate among unassigned nodes is an order of magnitude lower than that of nodes assigned to communities. 
Note that, while the MOSES algorithm will often assign nodes to multiple communities, it can also leave some nodes ``unassigned'' (\ie not assigned to any community at all).


We considered three sets of community finding experiments:
\begin{itemize}
\item Applying the MOSES algorithm to each individual monthly time step graph.
\item Applying the MOSES algorithm to the combined four month graph. This binary unweighted graph was created by aggregating the sets of nodes and edges present in the individual time step graphs.
\item Dynamic community finding on the four time step graphs. 
\end{itemize}


