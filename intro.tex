% !TEX root = bipartite.tex

\section{Introduction}

Social networks are becoming increasingly popular for modeling interactions
between nodes or individuals of the same kind (one-mode or unipartite) and there
is a large amount of work recently focusing on various aspects of social
networks. However some real-world networks have a more heterogeneous structure
where two kinds of individuals (or items) interact with each other with ties
formed between individuals of different types only. Bipartite networks
(two-mode) are a natural fit for these sort of systems as they are represented
as a graph with two disjoint sets of nodes and edges exists only between nodes
from different sets. There are many examples of complex networks that have a
bipartite structure, such as actor-movie networks, where actors are linked to
movies they played in \citep{guillaume2004bipartite}, authoring or collaboration
networks, where authors are linked to the paper they published
\citep{newman2001scientific1}, human sexual networks, consisting of men an women
\citep{liljeros2001web} and the metabolic networks between chemical reactions
and metabolites \citep{jeong2000large}. Even though there is a lot more focus on
social networks in the scientific community, research on bipartite or two-mode
networks has moved forward in the last couple of years. 

User-object networks are emerging as a special class of bipartite networks with
certain characteristics that are different than the relatively well studied
author-paper and actor-movie networks. Users interact continuously with objects,
based on their own selection and preference. This representation is appropriate
for modeling activities in web and e-commerce environments. For example the
social tagging, music listening activity or movie watching can be modelled as an
user-object network. In certain cases these networks can be also referred as
consumer-product networks where an edge links a consumer with a product when the
former buys or views that particular product \citep{huang2007analyzing}. 

One particular characteristic of user-object networks is the presence of a
significant number of highly popular objects reflected by the heavy tail
distribution of the objects degree \citep{shang10empirical}. The heavy tail is
believed to be formed through preferential attachment
\citep{barabasi2002evolution}, where users tend to interact more with objects
that are already popular. These popular objects have a significant effect on the
properties of both bipartite and projected networks, causing significant link
inflation in the latter. On the other side they are a poor indicator of users
interests, while unpopular objects are the best at describing common tastes
shared by users \citep{shang10empirical}. In this paper we are proposing a
method of (re)assigning weights to edges in a bipartite user-object network
based on the popular \emph{tf-idf} method that will reduce the effect of popular
objects while taking into account user's preferences. We validate the new method
by showing that both density and community structure of the projected users
network is improved compared to both the original projected network and the
network generated partially in a random fashion.

The remainder of the paper is structured as follows. Next section reviews the
state of the art literature on bipartite and user-object networks. In section 3
we provide a comprehensive analysis of user-object networks and their properties
and challenges, including density and clustering. Section 4 proposes the new
weighting scheme based on \emph{tf-idf}. In section 5 we validate the proposed
approach by comparing densities and modularities of the projected users network
to the random case. Finally, we conclude the paper in section 6 by summarizing
our findings and pointing out future research directions.
