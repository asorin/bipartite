%!TEX root = bipartite.tex

\section{Conclusion}

We analyzed the bipartite representation of interactions between users and
objects in web or e-commerce systems and shown that due to their scale such
user-object networks have certain characteristics and challenges differentiating
them from other bipartite networks such as collaboration or actor-movie
networks. We analyzed the structural properties of five real world user-object
networks and found a heavy tail degree distribution for objects. This
heterogeneity is responsible for hyperinflation of edges in the projected users
network resulting in very high density and diluted community structure in these
projections. Also, popular objects are connecting together a large number of
users, but they contain little or high level information about users interests.

In order to diminish the impact of higher degree objects we are proposing a new
weighting scheme based on the popular \emph{tf-idf} method used in information
retrieval and text mining. With \emph{tf-idf} the weight of interactions between
users and unpopular objects is amplified while popular objects will bring a
lower contribution to the weight of an edge. Filtering can be applied to keep
only the relevant edges with higher \emph{tf-idf} weights, starting from a
specific threshold. We used the proposed approach with five real world user-object
networks and demonstrate a decrease in density of both original and projected
networks.  We apply the Louvain method \citep{blondel2008fast} to find
communities in the projected users network and calculate the modularity of each
partition. We find that the quality of the community structure as measured by
modularity is significantly improved when compared to the projections of the
original networks. This improvement is also observed with similar partially
random networks where edges are filtered out randomly.
 
We proposed a simple and efficient method of assigning weights in a bipartite
user-object network that will decrease the impact of popular objects and improve
density and community structure. Current work can be extended to develop new
network-based recommender systems. Data used by such systems has a natural
bipartite structure where users are interacting with items in an online
environment. Our approach can also be applied to current community detection
methods for bipartite networks to improve both their scalability and the quality
of results. Further on, due to their simplicity, \emph{tf-idf} weights can be
easily adapted to suit temporal networks that evolve over time. These evolving
weighs can be monitored and analyzed to detect trends and patterns in
user-object event data.
